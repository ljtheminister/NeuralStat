\documentclass{article}
\usepackage[margin=1.0in]{geometry}

\begin{document}

\title{Decoding of reaching movements of monkeys}
\maketitle


\section{Introduction}

Ebullient research in the the field of brain-machine interfaces (BMIs) has 
choosing the optimal input signal source for a desired application.  

Single neuron action potentials or spikes, the signals with highest bandwith, are generally difficult to record after a few years after implantation due to the deterioration of the intracortical electrodes.  




Flint et. al (2012) 



\section{Experiment}
A rhesus monkey was trained to perform an eight-target, center-out task while holding onto a two-link manipulandum.  Eight 2 cm square outer targets spaced at 45$\,^{\circ}$ intervals around a circle of radius 10 cm.  Each trial begins with the illumination of the center target.  After a random hold time of 0.5-0.6 seconds, the center target disappears and a randomly selected outer target is illuminated, signaling the monkey to perform a reach movement.  The monkey must reach the outer target within 1.5 seconds and hold for a random time between 0.2 and 0.4 seconds to obtain a liquid reward.  

\section{Methodology}


\section{Results}

\section{Discussion}

\end{document}